\documentclass[12pt, letterpaper, titlepage]{article}

\usepackage{amsmath}
\usepackage{booktabs}
\usepackage{amsthm}
\usepackage{graphicx}
\usepackage[margin=1in]{geometry}
\usepackage{hyperref}
\hypersetup{colorlinks = true, linkcolor = blue, citecolor=blue, urlcolor = blue}
\usepackage{natbib}
\usepackage{enumitem}
\usepackage{setspace}

\usepackage[pagewise]{lineno}
%\linenumbers*[1]
% %% patches to make lineno work better with amsmath
\newcommand*\patchAmsMathEnvironmentForLineno[1]{%
 \expandafter\let\csname old#1\expandafter\endcsname\csname #1\endcsname
 \expandafter\let\csname oldend#1\expandafter\endcsname\csname end#1\endcsname
 \renewenvironment{#1}%
 {\linenomath\csname old#1\endcsname}%
 {\csname oldend#1\endcsname\endlinenomath}}%
\newcommand*\patchBothAmsMathEnvironmentsForLineno[1]{%
 \patchAmsMathEnvironmentForLineno{#1}%
 \patchAmsMathEnvironmentForLineno{#1*}}%

\AtBeginDocument{%
 \patchBothAmsMathEnvironmentsForLineno{equation}%
 \patchBothAmsMathEnvironmentsForLineno{align}%
 \patchBothAmsMathEnvironmentsForLineno{flalign}%
 \patchBothAmsMathEnvironmentsForLineno{alignat}%
 \patchBothAmsMathEnvironmentsForLineno{gather}%
 \patchBothAmsMathEnvironmentsForLineno{multline}%
}

% control floats
\renewcommand\floatpagefraction{.9}
\renewcommand\topfraction{.9}
\renewcommand\bottomfraction{.9}
\renewcommand\textfraction{.1}
\setcounter{totalnumber}{50}
\setcounter{topnumber}{50}
\setcounter{bottomnumber}{50}

\newcommand{\jy}[1]{\textcolor{blue}{JY: #1}}
\newcommand{\eds}[1]{\textcolor{red}{EDS: (#1)}}


\title{Time Series Length at Which Block Bootstrapping is Effective for Estimation of Variance}

\author{Mathew Chandy\\
%   \href{mailto:mathew.chandy@uconn.edu}
% {\nolinkurl{mathew.chandy@uconn.edu}}\\
  Jun Yan\\[1ex]
  Department of Statistics, University of Connecticut\\
}
\date{}

\begin{document}
\maketitle

\doublespace

\begin{abstract}
Block bootstrapping is a method that can be used for estimating a statistic of a time
series. It involves splitting a series into blocks (in order to account for the time
factor) and re-sampling the blocks to create many new bootstrapped time series.
This method becomes more effective as the length of the time series increases. 
The question for this study is how does one determine at what length the block
bootstrap method stops being an effective method to estimate a statistic of a time
series.

\bigskip
\noindent\sc{Keywords}:
block bootstrap;
\end{abstract}

\section{Introduction}
\label{sec:intro}

For this study, many block bootstrap simulations were conducted with R. At the base
level, we are block-bootstrapping an auto-regressive process with true mean 0.
Bootstrapping is the term for creating new samples of the same size by re-sampling from
the original sample with replacement. Many bootstrapped samples are used to create a
distribution of sampling means to estimate mean and variance. This works for samples
that are not dependent. However, for a time series such as an auto-regressive process,
a different procedure is required to account for the time dependence. In such a case,
the series is split into blocks that may overlap (moving block bootstrap) or may not
overlap (non-moving block bootstrap). These blocks are then re-sampled to create new
bootstrapped time series. 

In our experiment, we find the means of a 1000 block-bootstrapped time series, 
and create a 95 \% confidence interval of the means. We replicate this 1000 times, 
and record the proportion of confidence intervals that recover the true mean 
(coverage rate). We are effectively observing how successful the bootstrapping process
is at estimating the mean and variance. The key variable being observed was n, 
the length of the time series, or the size of the sample. It is known that as n
increases, block bootstrapping will become a less accurate method for estimation
(the coverage rates will decrease). The question is at what range of n values does this
start to become a problem. 

In this experiment, there are certain factors that are expected to affect the coverage
rates. As the AR coefficient (the time dependence) of the time series increases,
we expect the coverage rates of the confidence intervals to decrease.
The coverage rates are also affected by the size of the blocks - more specifically,
the ratio of the size of the blocks (l) to the size of the time series (L) - 
and whether or not they overlap.

So to solve the question of what n is necessary for effective block bootstrapping,
while still accounting for all these factors, the simulations were repeated for
multiple combinations of AR coefficient and l:L ratios. In addition, the coverage rates
for both the non-moving and moving methods were recorded.

\section{Literature Review}
\label{sec:litreview}




\section{Data}
\label{sec:data}




\section{Methods}
\label{sec:methods}




\section{Results}
\label{sec:results}




\section{Discussion}
\label{sec:discuss}



\bibliographystyle{chicago}
\bibliography{citations.bib}


\end{document}
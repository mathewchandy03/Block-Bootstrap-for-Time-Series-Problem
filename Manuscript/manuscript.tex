% Created by Bonita Graham
% Last update: December 2019 By Kestutis Bendinskas

% Authors: 
% Please do not make changes to the preamble until after the solid line of %s.

\documentclass[10pt]{article}
\usepackage[explicit]{titlesec}
\setlength{\parindent}{0pt}
\setlength{\parskip}{1em}
\usepackage{hyphenat}
\usepackage{ragged2e}
\RaggedRight

% % These commands change the font. If you do not have Garamond on your computer, you will need to install it.
% \usepackage{garamondx}
% \usepackage[T1]{fontenc}
% \usepackage{amsmath, amsthm}
% \usepackage{graphicx}
% \usepackage{hyperref}
% \usepackage{xcolor}
% \newcommand{\jy}[1]{\textcolor{orange}{JY: (#1)}}
% \usepackage{booktabs}

% This adjusts the underline to be in keeping with word processors.
\usepackage{soul}
\setul{.6pt}{.4pt}


% The following sets margins to 1 in. on top and bottom and .75 in on left and right, and remove page numbers.
\usepackage{geometry}
\geometry{vmargin={1in,1in}, hmargin={.75in, .75in}}
\usepackage{fancyhdr}
\pagestyle{fancy}
\pagenumbering{gobble}
\renewcommand{\headrulewidth}{0.0pt}
\renewcommand{\footrulewidth}{0.0pt}

% These Commands create the label style for tables, figures and equations.
\usepackage[labelfont={footnotesize,bf} , textfont=footnotesize]{caption}
\captionsetup{labelformat=simple, labelsep=period}
\newcommand\num{\addtocounter{equation}{1}\tag{\theequation}}
\renewcommand{\theequation}{\arabic{equation}}
\makeatletter
\renewcommand\tagform@[1]{\maketag@@@ {\ignorespaces {\footnotesize{\textbf{Equation}}} #1.\unskip \@@italiccorr }}
\makeatother
\setlength{\intextsep}{10pt}
\setlength{\abovecaptionskip}{2pt}
\setlength{\belowcaptionskip}{-10pt}

\renewcommand{\textfraction}{0.10}
\renewcommand{\topfraction}{0.85}
\renewcommand{\bottomfraction}{0.85}
\renewcommand{\floatpagefraction}{0.90}

% These commands set the paragraph and line spacing
\titleformat{\section}
  {\normalfont}{\thesection}{1em}{\MakeUppercase{\textbf{#1}}}
\titlespacing\section{0pt}{0pt}{-10pt}
\titleformat{\subsection}
  {\normalfont}{\thesubsection}{1em}{\textit{#1}}
\titlespacing\subsection{0pt}{0pt}{-8pt}
\renewcommand{\baselinestretch}{1.15}

% This designs the title display style for the maketitle command
\makeatletter
\newcommand\sixteen{\@setfontsize\sixteen{17pt}{6}}
\renewcommand{\maketitle}{\bgroup\setlength{\parindent}{0pt}
\begin{flushleft}
\sixteen\bfseries \@title
\medskip
\end{flushleft}
\textit{\@author}
\egroup}
\makeatother

% This styles the bibliography and citations.
%\usepackage[biblabel]{cite}
\usepackage[sort&compress]{natbib}
\setlength\bibindent{2em}
\makeatletter
\renewcommand\@biblabel[1]{\textbf{#1.}\hfill}
\makeatother
\renewcommand{\citenumfont}[1]{\textbf{#1}}
\bibpunct{}{}{,~}{s}{,}{,}
\setlength{\bibsep}{0pt plus 0.3ex}


\usepackage{float}

%%%%%%%%%%%%%%%%%%%%%%%%%%%%%%%%%%%%%%%%%%%%%%%%%

% Authors: Add additional packages and new commands here.  
% Limit your use of new commands and special formatting.

% Place your title below. Use Title Capitalization.
\title{Time Series Length at Which Block Bootstrapping is 
Effective for Estimation of a Statistic}

% Add author information below. Communicating author is indicated by an asterisk, 
% the affiliation is shown by superscripted lower case letter if several affiliations
% need to be noted.
\author{
Mathew Chandy*, Jun Yan \\ \medskip 
Department of Statistics, University of Connecticut, Storrs, CT  \\  \medskip 
Students: mathew.chandy@uconn.edu* \\
Mentor: jun.yan@uconn.edu 
}

\pagestyle{empty}

\begin{document}

% Makes the title and author information appear.
\vspace*{.01 in}
\maketitle
\vspace{.12 in}

% Abstracts are required.
\section*{abstract}

Block bootstrapping is a method that can be used for estimating a statistic of a time
series. It involves splitting a series into blocks (in order to account for the time
factor) and re-sampling the blocks to create many new bootstrapped time series.
This method becomes more effective as the length of the time series increases. 
The question for this study is how does one determine at what length the block
bootstrap method stops being an effective method to estimate a statistic of a time
series.

% Keywords are required.
\section*{keywords} 
Block Bootstrap

\vspace{.12 in}

% Start the main part of the manuscript here.
% Comment out section headings if inappropriate to your discipline.
% If you add additional section or subsection headings, use an asterisk * to avoid numbering. 


\section{Introduction}

For this study, many block bootstrap simulations were conducted with R. At the base
level, we are block-bootstrapping an auto-regressive process with true mean 0.
Bootstrapping is the term for creating new samples of the same size by re-sampling from
the original sample with replacement. Many bootstrapped samples are used to create a
distribution of sampling means to estimate mean and variance. This works for samples
that are not dependent. However, for a time series such as an auto-regressive process,
a different procedure is required to account for the time dependence. In such a case,
the series is split into blocks that may overlap (moving block bootstrap) or may not
overlap (non-moving block bootstrap). These blocks are then re-sampled to create new
bootstrapped time series. 

In our experiment, we find the means of a 1000 block-bootstrapped time series, 
and create a 95 \% confidence interval of the means. We replicate this 1000 times, 
and record the proportion of confidence intervals that recover the true mean 
(coverage rate). We are effectively observing how successful the bootstrapping process
is at estimating the mean and variance. The key variable being observed was n, 
the length of the time series, or the size of the sample. It is known that as n
increases, block bootstrapping will become a less accurate method for estimation
(the coverage rates will decrease). The question is at what range of n values does this
start to become a problem. 

In this experiment, there are certain factors that are expected to affect the coverage
rates. As the AR coefficient (the time dependence) of the time series increases,
we expect the coverage rates of the confidence intervals to decrease.
The coverage rates are also affected by the size of the blocks - more specifically,
the ratio of the size of the blocks (l) to the size of the time series (L) - 
and whether or not they overlap.

So to solve the question of what n is necessary for effective block bootstrapping,
while still accounting for all these factors, the simulations were repeated for
multiple combinations of AR coefficient and l:L ratios. In addition, the coverage rates
for both the non-moving and moving methods were recorded.

\section{Literature Review}





\section{Data}





\section{Methods}



\section{Results}



\section{Discussion}




\bibliographystyle{chicago}
\bibliography{citations.bib}

% \section*{acknowledgements}
% The authors thank the University of Connecticut and the UConn Department of Statistics.
%Note correct LaTeX quotations above. Do not use the " symbol, but rather double ` followed by double '


% The About the Student Author section is NOT optional.  Write a paragraph about the student; see previous journal editions for examples.
% If there is more than one student author, you must move the comment below.
\section*{about the student author}
%\section*{about the student authors}



% The Press Summary section is NOT optional.  Write a paragraph describing the paper in a manner suitable for the press; 
see previous journal editions for examples.
\section*{press summary}




\end{document}
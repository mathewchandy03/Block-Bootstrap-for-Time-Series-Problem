\documentclass[12pt]{article}
\usepackage[margin=1in]{geometry}
\usepackage{natbib}
\usepackage{hyperref}

\usepackage{listings}
\usepackage{rotating, graphicx}
\graphicspath{{./}, {./image/}}
\usepackage{booktabs, natbib}
% \usepackage{breakurl}
% \usepackage [english]{babel}
\usepackage{amsmath, amsbsy, amsthm, epsfig, epsf, psfrag, graphicx, 
amssymb, enumerate}
\usepackage{bm}
\usepackage{multirow, multicol}

\usepackage{color}
\definecolor{darkblue}{rgb}{0.1, 0.2, 0.6}
\usepackage{xcolor}
\newcommand{\jy}[1]{\textcolor{red}{JY: #1}}
\newcommand{\eds}[1]{\textcolor{blue}{(EDS: #1)}}
\newcommand{\mc}[1]{\textcolor{green}{(MC: #1)}}

\sloppy

% \usepackage{csquotes}
% \usepackage [autostyle, english = american]{csquotes}
% \MakeOuterQuote{"}

% \usepackage{bibentry}
\newenvironment{comment}%
{\begin{quotation}\noindent\small\it\color{darkblue}\ignorespaces%
}{\end{quotation}}


\begin{document}

\begin{center}
  {\Large\bf Response to the Comments}
\end{center}


% \jy{Spell check!}

\subsection*{Summary}

We thank the editor and the AE for the opportunity to revise the manuscript and
the two referees for their constructive comments. The manuscript has been
revised accordingly with the following notable changes:
\begin{enumerate}
\item 
  \begin{enumerate}
    \item 
    \item 
    \item 
    \item 
    \end{enumerate}
\item 
  \begin{enumerate}
  \item 
  \item 
  \item 
  \item 
  \end{enumerate}
\end{enumerate}



Point-by-point responses to the comments are as follows, with the
comments quoted in \emph{\color{darkblue} italic}.

\subsection*{To Referee 1}

\begin{comment}
1. It would be helpful to comment on the rationale for the proposed centered CI, 
especially in contrast to why a few others do not work for the log-1 
autocorrelation coefficient.
\end{comment}



\begin{comment}
2. For all the figures, would it be possible to put the 6 figures in a column in 
a single plot? It would also be helpful to make the figures a bit larger. Right 
now it is hard to compare across different methods regarding their comparative 
performance.
\end{comment}




\begin{comment}
3. The Appendix should be moved to the main text? Assessing the performance of 
various CI formulations for non-normal error is of central interest.
\end{comment} 



\begin{comment}
4. The authors might want to discuss in what circumstances “non-parametric” CI 
formulas would work better. It is a bit surprising that student’s t based 
formula works well under both normal and non-normal error structures.
\end{comment}

\mc{Not sure what they mean by "non-parametric" CIs, maybe the Percentile, BC,
and BCA, and Recentered CIs? In Figure 4, we showed that the autocorrelation in
the bootstrap samples is generally smaller than in the original sample. 
Percentile, BC, and BCA CIs perform better when the temporal dependence is low.
In these circumstances, "non-parametric" CIs would work better}

\begin{comment}
5. Page 3, 2nd line after “Standard Normal CI”: “p.168” does not seem to fit?
\end{comment}

\mc{Can we include the page number in the superscript somehow? Otherwise it does
not fit the journal's style.}

\subsection*{To Referee 2}

\begin{comment}
1. The authors say “....the standard CI is a table-based bootstrap…..”.  
What is meant by “table-based” here?  Is this just a reference to how critical 
values used to be listed in tables?  If that’s the case, I wouldn’t refer to 
this as a “table-based” interval. 
\end{comment}

\mc{The Efron book refers to such tables as "confidence intervals based on 
bootstrap 'tables'"}


\begin{comment}
2. Just a comment: The recentered percentile CI proposed here is an interesting 
interval to study. 
\end{comment}

\mc{Maybe we can add this to future studies}

\begin{comment}
3. The sections are numbered, which makes it awkward when the authors refer back 
to a section.  For instance, they refer to “...procedures described in Block 
Bootstrap CIS.”  This would be easier to refer to if the sections were numbered.  
(Is this a journal style choice?  If so, then you can ignore this comment.)
\end{comment}

\mc{It is a journal style choice}

\begin{comment}
4.  In Figure 1 (and all the other figures), the authors present confidence 
intervals for the real coverage rates.  The authors should specify what type of 
CI they are using here.  I assume the authors are using a Wald-type interval for 
this CI as this is for a proportion.  If that’s the case, it’s well known that 
the Wald-type interval has undercoverage when the true proportion is near 0 or 
1, which is the case in several of these scenarios, especially when phi is -0.4. 
\end{comment}

\mc{It is the Wald-type interval, should we just mention the undercoverage in
the design section of simulation and cite a source?}

\begin{comment}
5.  In all the figures, the y-axis is allowed to vary freely making it hard to 
compare one row directly to another.  I would suggest fixing the limits on the 
y-axis to make comparisons easier.  
\end{comment}

\mc{Because of the deterioration, we found it hard to fix the limits on the 
y-axis without making the other rows look bad}

\begin{comment}
6.  Is there a reason why the coverage for mu (in Figure 1) is better for 
negative autoregressive terms when compared with the positive autoregressive 
term? Can the authors give some intuition behind why that makes sense? 
\end{comment}



\begin{comment}
7.  The authors say “A smaller sample size is generally required to estimate a 
parameter for a sample with a negative phi versus a positive phi of the same 
magnitude.”  On a quick reading of this it sounds like the authors are 
recommending using a smaller sample size when phi is negative.  I believe what 
you are trying to say is that coverage rates are acceptable at smaller sample 
sizes when phi is negative versus when phi is positive (i.e. it takes larger 
sample sizes to get acceptable coverage when phi is positive).  Change this 
sentence to make it more clear. 
\end{comment}

\mc{addressed}

\begin{comment}
8.   In the Discussion section, the authors say “We know theoretically that the 
block bootstrap procedure will cover the parameter of a time series given an 
infinitely large sample.”  Essentially, you are stating that the estimator is 
consistent.  Is there a citation that you can point to that shows that this is 
true? 
\end{comment}



\begin{comment}
9.   The authors studied coverage rates of the different intervals as a function 
of sample size, but they only looked at one choice for the number of blocks.  
Why did the authors choose to not also look at how the number of blocks affects 
the coverage rates?
\end{comment}

\citep{ratick2009175}

\begin{comment}
10.   Why did the authors choose to only go as high as 0.4 (and as low as -0.4) 
for the autocorrelation parameter?  I would think that these correlations near 1 
(or -1) would be some of the more interesting cases to study, but they are 
completely ignored here.  Can the authors justify their choice to leave it out 
of this study?
\end{comment}

\mc{For the higher positive autocorrelations, a very large sample size is
required. We could try it for more extreme negative autocorrelations.}

\begin{comment}
11.    Last paragraph: The authors mention the “ABC and bootstrap-t intervals”.  
They do not give a citation for these intervals.   Add a citation. 
\end{comment}

\citep[p.160]{efron1993introduction}
\citep[p.188]{efron1993introduction}


\begin{comment}
12.    The last line of the manuscript offers one line about the drawbacks of 
the block bootstrap almost as an afterthought.  I think a discussion of the 
drawbacks should be expanded and added to the introduction as this is an 
important point that isn’t mentioned at all until the literal last line. 
\end{comment}

We are indebted to the reviewers for pointing out these vital discussion areas,
which have not only enriched the current paper but also charted a clear course
for our subsequent research endeavors.


\bibliographystyle{chicago}
\bibliography{citations}


\end{document}

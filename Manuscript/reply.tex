\documentclass[12pt]{article}
\usepackage[margin=1in]{geometry}
\usepackage{natbib}
\usepackage{hyperref}

\usepackage{listings}
\usepackage{rotating, graphicx}
\graphicspath{{./}, {./image/}}
\usepackage{booktabs, natbib}
% \usepackage{breakurl}
% \usepackage [english]{babel}
\usepackage{amsmath, amsbsy, amsthm, epsfig, epsf, psfrag, graphicx, 
amssymb, enumerate}
\usepackage{bm}
\usepackage{multirow, multicol}

\usepackage{color}
\definecolor{darkblue}{rgb}{0.1, 0.2, 0.6}

% \usepackage{xcolor}
\newcommand{\jy}[1]{\textcolor{red}{JY: #1}}
\newcommand{\tw}[1]{\textcolor{blue}{(TW: #1)}}
\newcommand{\sx}[1]{\textcolor{teal}{(SX: #1)}}

\sloppy

% \usepackage{csquotes}
% \usepackage [autostyle, english = american]{csquotes}
% \MakeOuterQuote{"}

% \usepackage{bibentry}
\newenvironment{comment}%
{\begin{quotation}\noindent\small\it\color{darkblue}\ignorespaces%
}{\end{quotation}}


\begin{document}

\begin{center}
  {\Large\bf Response to the Comments}
\end{center}


% \jy{Spell check!}

\subsection*{Summary}

We thank the editor and the AE for the opportunity to revise the manuscript and
the two referees for their constructive comments. The manuscript has been
revised accordingly with the following notable changes:
\begin{enumerate}
\item Newly added analyses:
  \begin{enumerate}
    \item chord plots visualizing the sectors' in- and out-connectivities in
      Section~3;
    \item key sector analysis with PageRank (PR) using export value as auxiliary
      information in Appendix~B;
    \item key sector analysis with hub and authority centrality in Appendix~C;
    \item dynamics of the similarity in the sector communities within each
      country and cross the two countries in Section~6.
    \end{enumerate}
\item Strengthened presentations:
  \begin{enumerate}
  \item elaborated motivation of using network tools for the comparison of
    sectoral structures in Section~1;
  \item expanded interpretation of directed assortativity coefficients in the
    context of input-output tables (IOTs) in Section~4;
  \item clarified discussion on the PR centrality using value added as
    auxiliary information in Section~5.
  \item completely rewritten discussions in Section~7.
  \end{enumerate}
\end{enumerate}
Additionally, we undertook meticulous reviews of the entire manuscript to ensure
clarity, coherence, and grammatical precision, aiming to present methods,
results, and discussions in a lucid manner.


Point-by-point responses to the comments are as follows, with the
comments quoted in \emph{\color{darkblue} italic}.

\subsection*{To AE}

\begin{comment}
1. Strengthen the motivation, i.e., explain how the China-Japan economy
comparison be better conducted using the methods consequently adopted in the
paper.
\end{comment}

As suggested, we have fortified our rationale for employing network analysis
in examining the sectoral structures of China and Japan. Specifically:
\begin{enumerate}
\item 
  Network analysis introduces advanced tools for innovative visualization.
\item
  Network analysis provides a deeper understanding of sectoral interdependencies,
  transcending the rudimentary summaries from the intermediate flow matrix that
  traditional input-output analysis offers. Notably, it encompasses metrics like
  directed strength distributions and weighted, directed assortativities, as well
  as sector-specific centrality measures.
\item
  Network analysis presents novel techniques for pinpointing key
  sectors and discerning sectoral clusters, which surpass the capabilities of
  conventional input-output analysis.
\end{enumerate}
To provide clearer exposition, we have thoroughly revised paragraph~3 of
Section 1, thus offering a compelling prelude to our analytical approach.


% \jy{Confirm that all the locations mentioned really contain the stated changes.}
% \tw{check and revise}

\begin{comment}
2. Perform sensitivity analysis of the PageRank-based results to the 
selection of the value of the tuning parameter; I also like the idea of using
the Kleinberg's hubs-and-authorities algorithm, which is a natural extension
of the PageRank idea; the idea of studying similarities over time is an
excellent one as well. 
\end{comment}

Concurring with the suggestion for sensitivity analyses, we have expanded our
examination to gauge the robustness of our key sector results obtained from the
extended PR approach.


\begin{table}[tbp]
\centering
\caption{The sectors with top~5 PR scores,
based on the sensitivity analysis of the tuning parameter at the levels
$\gamma = \{0.5, 0.85\}$ for China and Japan from 1995 to 2018
every three years. Total value added is used as auxiliary information.}
\label{tab:prsens}
\resizebox{\textwidth}{!}{
\begin{tabular}{cccccccccccccccccccc}
\toprule
\multirow{2}{*}{Country} & \multirow{2}{*}{Rank}
& \multicolumn{9}{c}{$\gamma = 0.5$} & \multicolumn{9}{c}{$\gamma = 0.85$} \\
\cmidrule(lr){3-11} \cmidrule(lr){12-20} 
& & 1995 & 1998 & 2001 & 2004 & 2007 & 2010 & 2013 & 2016 & 2018 & 
1995 & 1998 & 2001 & 2004 & 2007 & 2010 & 2013 & 2016 & 2018 \\ 
\midrule
China & 1 & 01 & 01 & 01 & 01 & 25 & 25 & 25 & 25 & 25 & 25 & 07 & 25 & 07 & 25 
& 25 & 25 & 25 & 25 \\ 
& 2 & 06 & 06 & 25 & 25 & 01 & 01 & 26 & 26 & 26 & 06 & 06 & 07 & 25 & 07 & 17 
& 20 & 06 & 17 \\ 
& 3 & 26 & 25 & 06 & 06 & 26 & 26 & 01 & 01 & 01 & 07 & 25 & 06 & 17 & 17 & 07 
& 06 & 17 & 40 \\ 
& 4 & 25 & 26 & 26 & 26 & 06 & 06 & 06 & 06 & 40 & 26 & 01 & 01 & 06 & 15 & 19 
& 26 & 20 & 26 \\ 
& 5 & 07 & 07 & 07 & 07 & 07 & 17 & 36 & 36 & 36 & 15 & 26 & 26 & 19 & 19 & 20 
& 17 & 07 & 07 \\ [1ex]
Japan & 1 & 26 & 26 & 26 & 26 & 26 & 26 & 26 & 26 & 26 & 26 & 26 & 26 & 26 & 26 
& 26 & 26 & 26 & 26 \\ 
& 2 & 25 & 25 & 37 & 37 & 37 & 37 & 37 & 37 & 37 & 25 & 25 & 25 & 20 & 20 & 42 
& 42 & 42 & 42 \\ 
& 3 & 37 & 37 & 25 & 42 & 42 & 42 & 42 & 42 & 42 & 42 & 42 & 42 & 42 & 42 & 20 
& 20 & 20 & 20 \\ 
& 4 & 42 & 42 & 42 & 25 & 25 & 25 & 25 & 25 & 25 & 20 & 20 & 20 & 25 & 37 & 37 
& 37 & 37 & 37 \\ 
& 5 & 36 & 36 & 36 & 36 & 20 & 20 & 20 & 20 & 20 & 37 & 37 & 37 & 37 & 25 & 25 
& 25 & 25 & 25 \\  
\bottomrule
\end{tabular}}
\end{table}

We first obtained PR results, also using value added as auxiliary information,
with $\gamma = 0.5$ and put them sides by side with those obtained with
$\gamma = 0.85$ in Table~\ref{tab:prsens}. Within this framework, $\gamma$
signifies the emphasis placed on the ranking from the IONs, while $1 - \gamma$
represents the emphasis put on the ranking of the auxiliary variable. The
widely-accepted value of $\gamma = 0.85$, as advocated by
\citet{page1998pagerank}, affirms a reasonable inclination towards the
importance of the IONs. The side-by-side comparison in Table~\ref{tab:prsens}
reinforces our stance that $\gamma = 0.85$ better aligns with our objectives. A
setting of $\gamma = 0.5$ seems to unduly accentuate the total value added,
potentially skewing the centrality dynamics. For comprehensive scrutiny, we have
incorporated this table and its pertinent discussion in our Supplementary
Material. To guide readers to this additional resource, we have integrated a
reference in the newly added paragraph~3 of Section~5.


Kleinberg's hubs-and-authorities algorithm~\citep{kleinberg1999authoritative}
presents an alternative approach to identifying key sectors in China and
Japan. The sectors identified via this approach deviate from those highlighted
by the PR method, underscoring the unique evaluative lenses inherent to each
technique. While both methods aim to identify central nodes, the criteria and
perspectives they adopt to discern centrality differ fundamentally. To afford
our readers an exhaustive understanding, we juxtaposed the outcomes of both
methods in the newly added Appendix~C.


Studying the similarities in the cluster structures over time is indeed an
excellent idea! To assess the resemblance between clusters, we adopted the
adjusted mutual information (AMI) metric~\citep{vinh2009information,
  vinh2010information}. Specifically, we have obtained the similarity of 
community structures between the two countries for the same year, as well as 
the dynamic similarity of community structures within a single country over 
time. Comprehensive explanations for these analyses have been included at the 
last paragraph of Section~6.


\begin{comment}
3. Improve the discussion section: better explain the assortativity analyses;
discuss the limitations of the work - this is quite important especially 
given that, as one of the reviewers notes, ``international input-output 
tables involving both countries are available but not used in this study''.
\end{comment} 

As suggested, we have completely rewritten the Discussion section with following
discussion points:
\begin{enumerate}
\item
  Methodological innovations in network analysis for assessing 
  disparities in the economic structures of China and Japan.
\item
  Abundance of centrality measures that can be used for key sector
  identification.
\item
  Availability of IOTs in national and international forms.
\item
  Potential followup investigation of China-Japan economic relations using
  multi-region IOTs.
\item
  Impacts of global shocks such as the 2018 global financial crisis and the
  Covid-19 pandemic.
\end{enumerate}

We acknowledge the need for a clearer interpretation of assortativity. In
response, we have added a new paragraph dedicated to the interpretation of
assortativity coefficients in terms of supplying sectors and receiving sectors.
As examples, we gave explicit interpretation of positive in-in assortativity and
negative out-out assortativity. See paragraph~2, Section~4.


We are grateful to the reviewers for highlighting the existence of international
IOTs encompassing both China and Japan. In the revised manuscript, we have
reviewed the World Input-Output Database (WIOD)~\citep{timmer2015illustrated}
and Asian Development Bank's Multiregional Input-Output Table database
(ADB-MRIO)~\citep{ADB2023iot}, explaining why they did not serve our purpose
well; see paragraph~3, Section~2. Further, the core emphasis of our paper is on
internal sectoral structures within each nation rather than the inter-country
connections. Analyzing the interconnectedness between sectors in China and Japan
remains a compelling avenue for future research, a prospect we delve into in
paragraph~4 of Section~7.


\begin{comment}
4. Expand on the insights produced; how do the reported findings compare to
those presented in the earlier-published literature? (this is important
since the authors state that the China-Japan economy comparison draws 
a lot of research attention. 
\end{comment}
  
We realized that our prior statement was not accurate.
While there is a significant volume of literature concentrating on the economies
of China and Japan independently, there is indeed a palpable gap in studies
directly juxtaposing the sectoral structures of these two major Asian
economies. To better reflect this reality, we have revisited and amended our
literature review in paragraph~2 of Section1. In the analysis, we undertook a
comparative exploration, accentuating both congruences and variances in relation
to the lone preceding study that touched on this theme, namely,
\citet{li2017examining}. This comparison is thoroughly elucidated in the
penultimate paragraph of Section~5.


\subsection*{To Referee 1}

\begin{comment}
1. One of my main concerns is contribution. The contribution of the paper is
 not methodological. Instead the contribution pretends to be in terms of a
better understanding of the differences between China and Japan's economies.
But the analysis done does not provide a deeper understanding different from
what one may expect. Both centrality ranking and clustering confirm what
was suspected or known. Therefore, the paper needs to make a deeper
network analysis and interpretation of results in order to show the
contribution and relevance of using network tools to analyze the
structure of economies.
\end{comment}


We appreciate the reviewer's concern on our contributions and would like to
clarify that our primary aim is to apply recent advancements in network analysis
tools, including some developed by our group, to analyze the IONs of China and
Japan's sectoral structures.


Regarding the observation that our contribution ``pretends'' to be in terms of a
better understanding of the differences between the two economies, we want to
assure that our intentions have been genuine and firmly rooted in academic
exploration. We acknowledge that sometimes, our analysis may seem to confirm
previously suspected ideas, but we see this as an essential validation in
research, further solidifying existing notions from a novel perspective.


In our literature review, we found a paucity of studies on this particular
topic, with a notable mention of \citet{li2017examining}. We would be grateful
for any additional references the reviewers could suggest to enhance our work.


To ensure clarity regarding our research contributions, the penultimate
paragraph in the introduction has been revised at large. See also our summary at
the beginning of this reply.


\begin{comment}
2. To study the structure of both economies authors use page rank as one of the
centrality measures. Why page rank? This measure requieres a tuning parameter
and therefore authors need to come up with an external measure for this. Many
times this is a subjective measure and ends up skewing the analysis.
\end{comment}


The PR method employed in our paper, which allows the inclusion of
auxiliary data, is an advancement of the standard PR traditionally used in
ION analyses \citep{cerina2015world}. This refined approach was selected
not only
because our group developed it and we were keen to assess its practical
application but also due to its alignment with our emphasis on economic
growth. Specifically, we aimed to embed the macroeconomic metric of value added
into the sectoral importance ranking. While we recognize in the revised
manuscript that this method could potentially bias the analysis, we also view
this as advantageous when the emphasis is on economic growth. This discussion
can be found in paragraph~4, Section~5.


We have also expanded our discussion to highlight the potential of alternative
centrality measures for key sector identification. See paragraph~2, Section~7.


\begin{comment}
3. Related to my previous point, why using value added as the tuning parameter
for page rank? As authors point out both economies are very different. Japan,
being a services economy is expected to have lower value added and a more
homogeneous distribution of value added across its most important sectors.
Comparably, China is a manufacturing economy with known industrial policy that
requieres certain amount of domestic value added, therefore is expected to have
higher value added and a more skewed distribution of it across sectors.
Importantly, we also have size of the economies; China is much bigger.
Consequently, using value added may skew things.
\end{comment}

% histogram vs bar chart

For our analysis, we used value added due to two primary
reasons. Firstly, given our focus on economic growth, value added serves as a
robust indicator of each sector's contribution to the overall economy. Secondly,
the IOTs readily provide this metric, ensuring its accessibility, and it has not
been utilized in the creation of the IONs.


The homogeneity of value added across sectors in China and in Japan piques
interest. To elucidate, we plotted bar charts of the value added data
for both countries in 1995 and 2018 in a newly added Figure~4. The plots suggest
marginal disparity in sectoral heterogeneity between the two countries; the
variance is 0.0261 and 0.0214 for China, and 0.0275 and 0.0278 for Japan.
Notably, as expected, sectors
contributing significantly to value added differ between these countries. This
variation underpins our motivation to integrate value added as the auxiliary
element in the PR metric. We have embedded this perspective in
paragraph~4, Section~5.


On the impact of the size of an economy on PR, we would like to clarify
that it is the composition of the auxiliary variable instead of its sizes that
matters. Specifically, $\lambda_i / \sum_{i \in V} \lambda_i$ is scale-free and
is only affected by the share of sector~$i$ in the whole economy. We have added
a clarification to end of paragraph~2, Section~5.

% There was no clarification before I added on Oct. 8

\begin{comment}
4. Therefore, I suggest two things. One, provide a sensitivity analysis and
observe how the ranking and conclusions change when using other macroeconomic
variables (or other type of variables) for the page rank tuning parameter.
Second, compute and use authority and hub scores which are a generalization of
eigenvector centrality for directed networks. They do not requiere a tuning
parameter therefore authors do not need to come up with a subjective external
measure to tune it in. These measures rank nodes according to the link
structure of the network and this is exactly what authors write that want to do.
\end{comment}

Excellent suggestions!


The choice of the auxiliary measure hinges upon the specific objectives 
of the researchers. In our case, we chose value added data as the auxiliary 
information because our focus centers on analyzing the sectoral structure, and 
value added data provides an effective reflection of the economic significance 
of each sector. Further, value added is conveniently available in the IOTs but
does not contribute to the construction of the IONs. Following the suggestion,
we performed a sensitivity analysis by using an alternative macroeconomic
variable, and reported the top sector results in Appendix~B. The results
referenced in the last paragraph, Section~5.


We agree that authority and hub scores have advantages of no need for subjective
auxiliary measures. In response, we have computed these scores and summarized
the results in Appendix~C. As expected, these results are different from ours
given the different focuses in the centrality measures. In fact, each of the
hundreds of existing centrality measures has its own rationale and
interpretation. In the paper, we kept the our results because our focus is on
economic growth and the recently proposed PR centrality with value added
as auxiliary variable meets our need. The results from authority and hub scores
are referenced in the last paragraph, Section~5. 


% obvious punctuation errors
\begin{comment}
5. Provide a better and more complete interpretation of assortativity. In
particular, about what is means assortativity in-in, out-out, in-out, positive
or negative in the context of input-output networks. Remember that links are
supply and demand of inputs. For example interprete and discuss what does it 
mean that there is positive assortativity for in-in but negative for out-out?
Discussion section is poor and not precise. Actually, it is not really a
discussion. Authors limit themselves to recapitulare main findings. Therefore I
highly recommend to provide a complete and more profound discussion section.
\end{comment}

We agree that the interpretation of assortativitity needed to be improved. In
response, we have added a new paragraph dedicated to the interpretation of
assortativity coefficients in terms of supplying sectors and receiving sectors.
As examples, we gave explicit interpretation
of positive in-in assortativity and negative out-out assortativity. See
paragraph~2, Section~4. 


We also agree that the discussion section in the last version did not suit the
journal style well. In response, we have completely rewritten the discussion
section covering the following:
\begin{enumerate}
\item
  Methodological innovations in network analysis for assessing 
  disparities in the economic structures of China and Japan.
\item
  Abundance of centrality measures that can be used for key sector
  identification.
\item
  Availability of IOTs in national and international forms.
\item
  Potential followup investigation of China-Japan economic relations using
  multi-region IOTs.
\item
  Impacts of global shocks such as the 2018 global financial crisis and the
  Covid-19 pandemic.
\end{enumerate}


\subsection*{To Referee 2}

\begin{comment}
1. The authors state that the economic structure comparison between China and
Japan ``has long been of interest''. But further justification and detailed
reasoning could be provided. The related development economics literature
should be surveyed.
\end{comment}

% quotes
% contractions
% proofread
% responsive

We have recognized that the phrase ``has long been of interest'' may not have
been entirely precise. While a plethora of studies exists on the economies of
both China and Japan, research that specifically zeroes in on comparing their
sectoral structures is relatively scant. Consequently, we have revised our
literature review on this topic, which can now be found in paragraph~2 of
Section~1.


We have also refined our justification for contrasting the sectoral structures
of Japan and China. This enhanced motivation can be found in the latter part of
paragraph~1 in Section~1, complete with pertinent references for further
clarity.

 
\begin{comment}
2. The economic interpretations of the network measures could be strengthened.
What are the economic implications of the findings? For example, the
substitutes (competition) and complements (cooperation) effects between China
and Japan from an international trade perspective can be seen in Zhu et al.
(2014) and from a global value chain perspective can be seen in Zhu et al.
(2018).
\end{comment}


We genuinely value the insightful feedback. Our core emphasis in this study
revolves around the internal sectoral structures of both China and Japan. To
reinforce the narrative and conclusions drawn from our findings, we have
enriched discussions across Sections 3 to 6. Notably:
\begin{enumerate}
\item
  We introduced chord plots in Section~3 to offer a more lucid visual
  representation of the intricate intersectoral connections.
\item
  Interpretations of the weighted, directed assortativity coefficients were
  added in paragraph~2, Section~4.
\item  
  Section~5 now includes an enhanced discussion on selection of auxiliary
  information in extended PR scoring; in discussions about the top 5 sectors,
  comparisons were made with established work by \citet{li2017examining}.
\item
  We have elucidated on the similarities in the community structures of sectors
  in the last paragraph of Section~6.
\end{enumerate}


While exploring the interconnected dynamics between China and Japan's sectors is
undoubtedly compelling, it remains beyond the immediate purview of this
research. Nevertheless, inspired by prior studies~\citep{zhu2014rise,
  zhu2018similarity}, we have alluded to the promising avenues of research in
this direction, leveraging multi-region IOTs, in paragraph~4 of Section~7.


\begin{comment}
3. The dynamics of the network measures are interesting. The authors could
apply a similarity measure (e.g., cosine similarity) for each year of the
clusters to give an overall picture for the comparison (e.g., are China and
Japan becoming more similar/dissimilar over time?). Then again, what might
be the economic development drivers behind the dynamics observed? 
\end{comment}


Thank you for the constructive feedback. Following your suggestion, we have
now incorporated the AMI as a metric to evaluate the similarity between
clusters. Our investigation compared two facets: the likeness of community
structures between China and Japan in a specific year and the temporal
similarity of community structures within each country. See detailed analysis in
the last paragraph, Section~6.


The evolving nature of community structures within both nations
presents a captivating narrative. It is striking to observe the subtle,
incremental modifications in Japan's sectoral makeup juxtaposed against China's
pronounced transformations over the years. The underlying reasons for these
pronounced shifts in China's sectoral layout are especially noteworthy. While
the scope of this paper does not extend to examining these catalysts in depth,
we concur that an exploration into the forces steering these dynamics would be a
compelling topic for subsequent research endeavors. A comment has been added to
the end of the last paragraph, Section~6.


\begin{comment}
4. The limitations of the current approach should be discussed further. For
example, the current study uses national input-output tables. But
international input-output tables involving both countries are also
available. What would be the pros and cons using alternative IO tables
(national versus international)? The most recent economic issues (e.g.,
Covid) cannot be discussed as the IO tables are only available until 2018.
Any remedy or thoughts on this? 
\end{comment}

% \jy{are these references cited in the paper? Fix the
% inconsistencies}\tw{delete the ucited.}.


We have extensively revised the discussion section, Section~7, to provide a
holistic overview of our study's limitations and potential avenues for future
exploration. To specifically address the concerns raised:
\begin{enumerate}
\item
  We discuss alternative IOT databases in paragraph~3.
\item
  The potential advantages of employing multi-region IOTs are highlighted in
  paragraph~4.
\item
  The implications of global shocks, including the recent ramifications of the
  Covid-19 pandemic, are pointed out as a promising direction in paragraph~5.
\end{enumerate}
We are indebted to the reviewers for pointing out these vital discussion areas,
which have not only enriched the current paper but also charted a clear course
for our subsequent research endeavors.


\bibliographystyle{chicago}
\bibliography{cjiot}


\end{document}

\documentclass[12pt]{article}
\usepackage[margin=1in]{geometry}
\usepackage{natbib}
\usepackage{hyperref}

\usepackage{listings}
\usepackage{rotating, graphicx}
\graphicspath{{./}, {./image/}}
\usepackage{booktabs, natbib}
% \usepackage{breakurl}
% \usepackage [english]{babel}
\usepackage{amsmath, amsbsy, amsthm, epsfig, epsf, psfrag, graphicx, 
amssymb, enumerate}
\usepackage{bm}
\usepackage{multirow, multicol}

\usepackage[dvipsnames]{color}
\definecolor{darkblue}{rgb}{0.1, 0.2, 0.6}

\newcommand{\jy}[1]{\textcolor{red}{JY: #1}}
\newcommand{\eds}[1]{\textcolor{blue}{(EDS: #1)}}
\newcommand{\mc}[1]{\textcolor{green}{(MC: #1)}}

\sloppy

% \usepackage{csquotes}
% \usepackage [autostyle, english = american]{csquotes}
% \MakeOuterQuote{"}

% \usepackage{bibentry}
\newenvironment{comment}%
{\begin{quotation}\noindent\small\it\color{darkblue}\ignorespaces%
}{\end{quotation}}


\begin{document}

\begin{center}
  {\Large\bf Response to the Comments}
\end{center}

We extend our gratitude to the editor and the Associate Editor for
granting us the opportunity to revise this manuscript. We also want to
express our appreciation to the two referees for their valuable and
constructive comments. The reviewers have highlighted crucial
discussion areas that have not only enhanced the quality of this paper
but have also set a clear direction for our future research
endeavors.


The manuscript has been
revised accordingly with the following notable changes:
\begin{enumerate}
\item Clarifications of design and more detailed examination of results;
\item Reorganization of paragraphs on the Simulation Study, including moving the
Appendix on the Exponential Marginal Distribution Study to the main text;
\item Further discussion of the justification, advantages, and implications of
the Recentered Percentile CI;
\item Additional simulation study using block size 
$l = \lceil 2n^{1/3} \rceil$ in the Appendix.
\end{enumerate}


Point-by-point responses to the comments are as follows, with the
comments quoted in \emph{\color{darkblue} italic and blue}.

\subsection*{To Referee 1}

\begin{comment}
1. It would be helpful to comment on the rationale for the proposed centered CI, 
especially in contrast to why a few others do not work for the log-1 
autocorrelation coefficient.
\end{comment}


Thank you for this suggestion. The motivation behind the proposed
recentered CI is that for the autocorrelation parameter in the
simulation studies, the centers of the existing CIs appeared to be a
biased estimator of the target while the spreads of the CIs still
provide good uncertainty measures. This has now been added in the
Recentered Percentile CI paragraph.


\begin{comment}
2. For all the figures, would it be possible to put the 6 figures in a column in 
a single plot? It would also be helpful to make the figures a bit larger. Right 
now it is hard to compare across different methods regarding their comparative 
performance.
\end{comment}


Thank you for the suggested idea. We have reformatted all of our coverage rate 
plots accordingly, which has saved a lot of space!

\begin{comment}
3. The Appendix should be moved to the main text? Assessing the performance of 
various CI formulations for non-normal error is of central interest.
\end{comment} 

Agreed. The text highlighting the simulation study with an exponential marginal
distribution from the Appendix has been moved to the Simulation Study 
section. 


\begin{comment}
4. The authors might want to discuss in what circumstances “non-parametric” CI 
formulas would work better. It is a bit surprising that student’s t based 
formula works well under both normal and non-normal error structures.
\end{comment}

Thank you for the comment. We expect the CIs that 
are based on bootstrap ``tables" to depend on the 
asymptotic distribution of the point estimator. This asymptotic
distribution depends more on the sample size than on the marginal
distribution of the time series. So we expect such CIs to 
have similar performance for different marginal distributions when the
sample sizes is large.  The percentile-based CIs (Percentile, BC, BCA,
Recentered Percentile) are not necessarily expected to perform
better under non-normal marginal distributions. The student's t and
and normal-based CIs are only noticeably different when the number of
blocks is smaller than 20. We have added a discussion on this point in
the first paragraph of the Marginal Unit Exponential
Distribution subsection.


\begin{comment}
5. Page 3, 2nd line after “Standard Normal CI”: “p.168” does not seem to fit?
\end{comment}

Because of the bib style constraints for AJUR, we have removed page numbers in 
citations.

\subsection*{To Referee 2}

\begin{comment}
1. The authors say “....the standard CI is a table-based bootstrap…..”.  
What is meant by “table-based” here?  Is this just a reference to how critical 
values used to be listed in tables?  If that’s the case, I wouldn’t refer to 
this as a “table-based” interval. 
\end{comment}

Thank you for the comment.  This terminology was used because 
\citet{efron1993introduction} refers to such CIs as
``confidence intervals based on 
bootstrap `tables'", which essentially means CIs based on asymptotic 
distributions with estimated asymptotic variances (standard error), as opposed
to CIs that are based on empirical percentiles.
We have replaced all instances of the phrase 
``table-based bootstrap CIs" to ``confidence intervals based on bootstrap 
`tables'" for the sake of clarity.

%The standard CI is classified by \citet{efron1993introduction} 
%as a confidence interval based on bootstrap ``tables".
%Like the standard normal interval, the
%Student's $t$ CI is classified by \citet{efron1993introduction} 
%as a confidence interval based on bootstrap ``tables".


\begin{comment}
2. Just a comment: The recentered percentile CI proposed here is an interesting 
interval to study. 
\end{comment}

We agree. We have added a comment in the Discussion underlining this CI's
potential for future investigations.

\begin{comment}
3. The sections are unnumbered, which makes it awkward when the
authors refer back
to a section.  For instance, they refer to “...procedures described in Block 
Bootstrap CIS.”  This would be easier to refer to if the sections were numbered.  
(Is this a journal style choice?  If so, then you can ignore this comment.)
\end{comment}

We found unnumbered sections inconvenient too, but unfortunately, 
this is a journal style choice.

\begin{comment}
4.  In Figure 1 (and all the other figures), the authors present confidence 
intervals for the real coverage rates.  The authors should specify what type of 
CI they are using here.  I assume the authors are using a Wald-type interval for 
this CI as this is for a proportion.  If that’s the case, it’s well known that 
the Wald-type interval has undercoverage when the true proportion is near 0 or 
1, which is the case in several of these scenarios, especially when phi is -0.4. 
\end{comment}

Thank you for pointing this out. We tried both Wald intervals and exact 
Clopper-Pearson intervals \citep{clopper1934use}, the latter of which
as implemented in the R package \texttt{PropCIs}.\citep{PropCIs}
No major difference between the two was found, which is expected as we
have a large number replication (10,000). Nonetheless, we have
switched to reporting Clopper-Pearson intervals to avoid the potential
undercoverage and discussed this in the third paragraph of the Marginal 
Standard Normal 
Distribution subsection of the Design section.
%\jy{Quanlity control the bib entries.}\mc{addressed}

\begin{comment}
5.  In all the figures, the y-axis is allowed to vary freely making it hard to 
compare one row directly to another.  I would suggest fixing the limits on the 
y-axis to make comparisons easier.  
\end{comment}

As suggested by the other reviewer, we now overlay the comparison of 6
CIs within the same panels. Each row reports the results for one
parameter, with a shared scale on the y-axis. The number of figures
have been greatly reduced and the comparison made easier.

\begin{comment}
6.  Is there a reason why the coverage for mu (in Figure 1) is better for 
negative autoregressive terms when compared with the positive autoregressive 
term? Can the authors give some intuition behind why that makes sense? 
\end{comment}

Thank you for this comment.
A possible explanation for this is that if a stationary series has a positive
autocorrelation, the effective sample size is decreased, whereas is a
series has a negative autocorrelation, theeffective sample size is
increased.\citep{geyer2011introduction}
Additionally, this seems to have a 
greater effect on the the estimation of the location parameter versus that
of the scale parameter or temporal dependence parameter. We have included a 
comment at the end of the paragraph on estimation of~$\mu$.

\begin{comment}
7.  The authors say “A smaller sample size is generally required to estimate a 
parameter for a sample with a negative phi versus a positive phi of the same 
magnitude.”  On a quick reading of this it sounds like the authors are 
recommending using a smaller sample size when phi is negative.  I believe what 
you are trying to say is that coverage rates are acceptable at smaller sample 
sizes when phi is negative versus when phi is positive (i.e. it takes larger 
sample sizes to get acceptable coverage when phi is positive).  Change this 
sentence to make it more clear. 
\end{comment}

Thank you for pointing this out. The sentence has been rewphrased as
suggested.

\begin{comment}
8.   In the Discussion section, the authors say “We know theoretically that the 
block bootstrap procedure will cover the parameter of a time series given an 
infinitely large sample.”  Essentially, you are stating that the estimator is 
consistent.  Is there a citation that you can point to that shows that this is 
true? 
\end{comment}

Thank you for drawing this to our attention. A citation \citep{calhoun2018} has
now been provided for that comment.

\begin{comment}
9.   The authors studied coverage rates of the different intervals as a function 
of sample size, but they only looked at one choice for the number of blocks.  
Why did the authors choose to not also look at how the number of blocks affects 
the coverage rates?
\end{comment}

Thank you for the comment. We have now repeated the same simulation study with 
block size as an
additional experimental factor. The optimal order of block size is 
$\lceil n^{1/3} \rceil$, so we tried the same simulations using 
$l = \lceil n^{1/3} \rceil$ and $l = \lceil 2n^{1/3} \rceil$. We have included
an Appendix detailing these results, as well as a discussion referring to these
results at the end of the Simulation Study Results section.


\begin{comment}
10.   Why did the authors choose to only go as high as 0.4 (and as low as -0.4) 
for the autocorrelation parameter?  I would think that these correlations near 1 
(or -1) would be some of the more interesting cases to study, but they are 
completely ignored here.  Can the authors justify their choice to leave it out 
of this study?
\end{comment}

Thank you for the comment. We only
used serial dependences as strong as 0.4, because we only seek to 
establish the 
general trend as the strength of the autocorrelation 
increases, and how it varies depending on the sign of the autocorrelation and 
the parameter of interest. This justification has been added to the first
paragraph of the Normal section of the Design 
section. We have also added a deliberation (third paragraph in the Discussion) 
regarding what would be expected for higher dependence levels.


\begin{comment}
11.    Last paragraph: The authors mention the “ABC and bootstrap-t intervals”.  
They do not give a citation for these intervals.   Add a citation. 
\end{comment}

We apologize for the omission. These intervals are discussed in 
\citet{efron1993introduction}. A citation is now provided in the last paragraph.

\begin{comment}
12.    The last line of the manuscript offers one line about the drawbacks of 
the block bootstrap almost as an afterthought.  I think a discussion of the 
drawbacks should be expanded and added to the introduction as this is an 
important point that isn’t mentioned at all until the literal last line. 
\end{comment}

Thank you for the comment. We have included a discussion of problems 
with block bootstrap as the second paragraph in the introduction, while also 
highlighting its advantages
which motivated this investigation into their CI performance. This
last few sentences of the manuscript have also been rewritten to call for
understanding of the subpar performance of the existing
percentile-based CIs for the autocorrelation parameter and situations
where the proposed CI should be recommended.


\bibliographystyle{chicago}
\bibliography{citations}


\end{document}
